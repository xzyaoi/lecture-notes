\documentclass{article}
\usepackage[utf8]{inputenc}
\usepackage{graphicx}
\usepackage{amsmath, amsthm, amssymb}
\usepackage{multicol}
\usepackage{svg}
\usepackage{caption}
\usepackage{vmargin}
\usepackage[hidelinks]{hyperref}
\usepackage[utf8]{inputenc}
\usepackage{epigraph}

\theoremstyle{definition}
\newtheorem{defi}{Definition}[subsection]
\newtheorem{theorem}{Theorem}[subsection]
\newtheorem{prop}{Proposition}[subsection]
\newtheorem{lemma}{Lemma}[subsection]

\title{Probability Models and Axioms}
\author{Xiaozhe Yao\footnote{https://yaonotes.org/lecture-notes}}
\date{06 Feb 2020}
\begin{document}

\maketitle

\section{Basic Concepts}

\begin{defi}
\textbf{Sample Space} is the set of all possible outcomes of an experiment. The elements of the set should have certain properties: \begin{itemize}
    \item \textit{Mutually Exclusive}. At the end of the experiment, there can only be one of the outcomes that has happened.
    \item \textit{Collectively Exhaustive}. At the end of the experiment, we should always be able to point to \textbf{exactly} one of the possible outcomes as the outcome that occurred.
    \item \textit{At the right Granularity}. The sample space that we pick should give what we're interested in.
\end{itemize}
\end{defi}

\end{document}