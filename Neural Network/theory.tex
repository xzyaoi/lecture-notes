\documentclass[12pt,openany]{book}
\setlength{\headheight}{15pt}
\usepackage{amsmath, amsthm, amssymb}
\usepackage{mdframed}
\usepackage{lipsum}
\usepackage{framed}

\newmdtheoremenv{thm}{Theorem}[section]
\newmdtheoremenv{exmp}{Example}[section]
\newtheorem*{tips}{Tips}

\theoremstyle{definition}
\newtheorem{defi}{Definition}[section]
\newtheorem*{lemma}{Lemma}
\newtheorem{cor}{Corollary}[section]
\newenvironment{soln}{\begin{proof}[Solution]}{\end{proof}}
\newenvironment{comment}{\begin{proof}[Comment]}{\end{proof}}
\newenvironment{motivation}{\begin{proof}[Motivation]}{\end{proof}}
\newtheorem{psol}{Problem}[section]
\newtheorem{prob}{Problem}[section]
\newtheorem{hint}{Hint}[section]
\usepackage{amsthm,amssymb,amsmath}
\theoremstyle{definition}
\setcounter{section}{1}
\newtheorem*{case}{Example}
\newtheorem{tip}{Tip}[section]
\usepackage{titlesec}
\definecolor{shadecolor}{RGB}{180,180,180}
\titleformat{\chapter}[display]
{\normalfont\bfseries\filcenter}
{\LARGE\thechapter}
{1ex}
{\titlerule[2pt]
\vspace{2ex}%
\LARGE}
[\vspace{1ex}%
{\titlerule[2pt]}]


\usepackage{cancel}
\usepackage[margin=4cm]{geometry}
\usepackage{hyperref}
\usepackage{fancyhdr}
\pagestyle{fancy}
\fancyhead{}
\fancyfoot{}
\lhead{Neural Network}
\chead{Xiaozhe Yao}
\rhead{\thepage}
\newenvironment{dedication}
    {\vspace{6ex}\begin{quotation}\begin{center}\begin{em}}
    {\par\end{em}\end{center}\end{quotation}}
    
\newcommand{\HRule}{\rule{\linewidth}{0.5mm}} % Defines a new command for the horizontal lines, change thickness here
\title{Algebra}
\begin{document}
% Center everything on the page
 

%----------------------------------------------------------------------------------------
%	TITLE SECTION
%----------------------------------------------------------------------------------------
\begin{center}
\HRule \\[0.4cm]
{ \huge \bfseries Theoretical Neural Network}\\[0.4cm] % Title of your document
\HRule \\[1.5cm]
\begin{minipage}{0.4\textwidth}
\begin{flushleft} \large
\emph{Authors}\\
Xiaozhe Yao \newline
www.yaonotes.org \newline

\end{flushleft}
\end{minipage}
~
\begin{minipage}{0.4\textwidth}
\begin{flushright} \large

\end{flushright}
\end{minipage}\\[0.5cm]
\end{center}

\chapter{Introduction}

\chapter{Linear Classifier}

\chapter{FeedForward Network}

\section{Non-Linear Classifier}

Linear classifier works well for linear classifiable datasets. However, there are numerous datasets that cannot be well classified by linear classifier, i.e. they are not linear separable. For example, Let's consider XOR.

$X=\{(0,0),(0,1),(1,0),(1,1)\}$ while,
$Y=\{0,1,1,0\}$

As a human, we can easily conclude and predict the value in $Y$. It is $1$ if $a_{x_i} = b_{x_i}$ or $0$ otherwise. However if you use a linear classifier, you can get a maximum $75\%$ accuracy.

\textit{Proof.} $\hfill \square$

There are several ways to handle the problem. One is by introducing transformations.

Another idea is by introducing non-linear classifiers.

$X=\{0,0,0,1\}$

\section{Intuition}


\section{}

\begin{snugshade*}
\noindent\textsc{Extra Curricular Achievements}
\end{snugshade*}


\end{document}