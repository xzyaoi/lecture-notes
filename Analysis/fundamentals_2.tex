\documentclass{article}
\usepackage[utf8]{inputenc}
\usepackage{graphicx}
\usepackage{amsmath, amsthm, amssymb}
\usepackage{multicol}
\usepackage{svg}
\usepackage{caption}
\usepackage{vmargin}
\usepackage[hidelinks]{hyperref}

\theoremstyle{definition}
\newtheorem{defi}{Definition}[section]
\newtheorem{theorem}{Theorem}[section]
\newtheorem{proposition}{Proposition}[section]
\newtheorem{lemma}{Lemma}[section]

\title{Fundamentals in Analysis II}
\author{Xiaozhe Yao\footnote{https://yaonotes.org/lecture-notes/maths/analysis/}}
\date{20 Feb 2020}
\begin{document}

\maketitle
\section{Differential Calculation}
\subsection{Differentiability}
\begin{defi}
The function $f:M\to Y$ is called differentiable at point $x_0\in M$, if the limit $\lim_{x\to x_0}\frac{f(x)-f(x_0)}{x-x_0}$ in $Y$ exists. If $f$ is differentiable at point $x_0$, we write it as: $f'(x_0):=Df(x_0) := \frac{df}{dx}(x_0) := \lim_{x\to x_0}\frac{f(x)-f(x_0)}{x-x_0} := \lim_{h\to 0}\frac{f(x_0+h)-f(x_0)}{h}$, and we call it the derivative of $f$ in $x_0$.
\end{defi}
\begin{prop}
The following statements are equivalent:
\begin{itemize}
    \item $f$ is differentiable in $x_0$.
    \item There exists a $c\in Y$ such that $\lim_{x\to x_0}\frac{f(x)-f(x_0)-c\cdot (x-x_0)}{x-x_0}=0$.
    \item There exists a $c\in Y$, and a map $r: M\to Y$, that are continuous in $x_0$ and $r(x_0)=0$, so that we have $f(x)=f(x_0)+c(x-x_0)+r(x)(x-x_0), \forall x\in M$.
\end{itemize}
\end{prop}
\textit{Note:} In case $2$, $3$, $c$ is the derivative of $f$ in $x_0$.
\begin{proof} We prove this one by one.
\begin{itemize}
    \item $1\implies 2$:It is trivial, by letting $c=f'(x_0)$.
    \item $2\implies 3$:
    \item $3\implies 1$:
\end{itemize}
\end{proof}
\begin{theorem}
\textbf{Basic Calculation Laws}
We have the following calculation laws:
\begin{itemize}
    \item $f: \mathbb{C}\to\mathbb{C}, f(z)=z^{n}, n\in\mathbb{N}_{\geq1}$ is differentiable for all $z\in\mathbb{C}$ and we have $f'(z)=nz^{n-1}$.
    \item $f:\mathbb{C}\to\mathbb{C}, f(z):=e^{z}$ is differentiable for all $z\in\mathbb{C}$ and $f'(z)=e^z$.
    \item $f: \mathbb{R}-\{0\}\to\mathbb{R}, f(x)=\frac{1}{x}$ is differentiable for any $x\in\mathbb{R}-\{0\}$, and $f'(x)=-\frac{1}{x^2}$.
    \item $f: \mathbb{R}\to\mathbb{R}^{2}, f(x)=(x, x^2)$ 
\end{itemize}
\end{theorem}
\end{document}

