\documentclass{article}
\usepackage[utf8]{inputenc}
\usepackage{graphicx}
\usepackage{amsmath, amsthm, amssymb}
\usepackage{multicol}
\usepackage{svg}
\usepackage{caption}
\usepackage{vmargin}
\usepackage[hidelinks]{hyperref}
\usepackage[utf8]{inputenc}
\usepackage{epigraph}

\theoremstyle{definition}
\newtheorem{defi}{Definition}[section]
\newtheorem{theorem}{Theorem}[section]
\newtheorem{proposition}{Proposition}[section]
\newtheorem{lemma}{Lemma}[section]

\title{Basic Concepts in Analysis}
\author{Xiaozhe Yao\footnote{https://yaonotes.org/lecture-notes}}
\date{20 Nov 2019}
\begin{document}

\maketitle
\section{Relation}

\begin{defi}
\textbf{Relation} Let $M$ be a set, a relation on $M$ is a subset $R$ of $M \times M$. It can be written as $x \sim y$ or $x\sim_{R}y$. It is called $x$ is in relation with $y$ with the respects of $R$, if $(x,y)\in R$ applies.
\end{defi}

\begin{defi}
\textbf{Equivalent Relation} Let M be a set, the relation $R$ over $M \times M$ is called equivalent relation, if: 
\begin{itemize}
    \item \textit{Reflexive} $x \sim_R x$. $\forall x\in M$.
    \item \textit{Symmetric} $x \sim_R y \implies y \sim_R x$, $\forall x, y \in M$.
    \item \textit{Transitive} $x \sim_R y \bigwedge y \sim_R z \implies x \sim_R z$. $\forall x, y, z \in M$.
\end{itemize}
\end{defi}

\begin{defi}
\textbf{Equivalent Class} Let $R$ be an equivalent relation on a set $M$, for $m \in M$, the equivalent class of $m$ is given by $[m] := \{x \in M: x\sim_R m\}$. Literally, the equivalent class is such a set that all the element are in a equivalent relation with the given $m$.

\textbf{Representative and Quotient Set}Each element $x \in [m]$ is called a representative of the equivalent class $[m]$. All the equivalent classes consists of the Quotient Set, i.e. $M / \sim := \{[m]: m\in M\}$.

\end{defi}

\begin{theorem}
\textbf{Disjoint Composition} Let $R$ be an equivalent relation on a set $M$. Then we have $[x] \cap [y] \neq \phi \implies [x]=[y]$. It follows that $M=\bigcup_{[m]\in M/\sim}[m]$ is a disjoint composition of the given $M$.
\end{theorem}

\begin{defi}
\textbf{Ordered Relation} A relation $R$ on $M$ is called ordered relation, if: 
\begin{itemize}
    \item $x \sim_{R} x$, $\forall x \in M$. (Reflexive)
    \item $x \sim_{R} y \bigwedge y \sim_{R} z \implies x \sim_{R} z$, $\forall x, y, z \in M$. (Transitive)
    \item $x \sim_{R} y \bigwedge y \sim_{R} x \implies x = y$, $\forall x, y \in M$. (Identity)
\end{itemize}

If a relation is an ordered relation, then it could be written as $x \preceq y$ instead of $x \sim_{R} y$.

A set $M$, along with an ordered relation $\preceq$ is called (partially) ordered set. If $\forall x,y \in M$, we have $x \preceq y \lor y \preceq x$, then it is called totally ordered set.

For example, $(\mathbb{R},\leq)$ is a totally ordered set. For any set $M$, $(P(M), \subset)$ is an ordered set, but generally not a totally ordered set.

\end{defi}

\begin{defi}
\textbf{Maximal and Minimal} Let $(M, \preceq)$ be an ordered set, and $A \subset M, A \neq \phi$. $m\in A$ is called the \textbf{maximal} if $\forall x, (x \in A)\bigwedge(m\preceq x) \implies x = m$. Respectively, $m\in A$ is called minimal if $\forall x, (x \in A)\bigwedge(x\preceq m) \implies x = m$.

In general, the maximal and minimal elements do not have to be exist or unique. However, if the relation is totally ordered, then they are unique if they exist.

\begin{proof}

Assume the relation $R$ is totally ordered relation, and $x_1, x_2$ are two maximal. Then either $x_{1} \preceq x_{2}$ or $x_{2} \preceq x_{1}$ applies. We assume that $x_{1} \preceq x_{2}$ applies, then since $x_{1}$ is the maximal, we have $x_{2}=x_{1}$. Similarly, we have this applied on minimal situations. 
\end{proof}
\end{defi}

\begin{defi}
\textbf{Upper and Lower Bounds} Let $(M, \preceq)$ an ordered set, $A \subset M, A \neq \phi$. $m \in M$ is called an \textbf{upper bound} for $A$ if $x \preceq m$, $\forall x\in A$. $m \in M$ is called a \textbf{lower bound} for $A$ if $m \preceq x$, $\forall x \in A$. The upper (lower) bounds indicate the upper (lower) limit.

\textit{Differences between Maximal/Minimal and Bounds}: Bounds do not have to be in the set.

\end{defi}

\begin{defi}
\textbf{Supremum and Infimum} Let $(M, \preceq)$ an totally ordered set and $A\subset M$ an upper limited set. Let $B=\{x\in M: x $ is upper bounds of $A\}$. Assume that $B \neq \phi$, and B contains a minimal element $b \in B$, then $b$ is called supremum of A: $b = sup(A)$. \textbf{Note} that $sup(A)$ is unique, if it exists. In general, the supremum $b = sup(A)$ does not have to belong to the set $A$ (by definition, $sup(A)$ belongs to the set of the upper bounds of $A$.
\end{defi}

\section{Map and Function}

\section{Number}

\subsection{Natural Number}

\begin{defi}
\textbf{Peano Axiom} The natural numbers form a set $\mathbb{N}$ with a starting element $0$ and a map $v: \mathbb{N}\to \mathbb{N}$ with the following characters:

\begin{itemize}
    \item $v$ is injective.
    \item $0 \not\in Range(v)$
    \item Induction Principle applies, i.e. $n\in A \implies v(n) \in A$.
\end{itemize}
\end{defi}

\end{document}



