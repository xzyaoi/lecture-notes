\documentclass{article}
\usepackage[utf8]{inputenc}
\usepackage{graphicx}
\usepackage{amsmath, amsthm, amssymb}
\usepackage{multicol}
\usepackage{svg}
\usepackage{caption}
\usepackage{vmargin}
\usepackage[hidelinks]{hyperref}
\usepackage[utf8]{inputenc}
 \usepackage{epigraph}

\theoremstyle{definition}
\newtheorem{defi}{Definition}[section]
\newtheorem{theorem}{Theorem}[section]
\newtheorem{proposition}{Proposition}[section]
\newtheorem{lemma}{Lemma}[section]

\title{Semantic Web Engineering}
\author{Xiaozhe Yao\footnote{https://yaonotes.org/lecture-notes}}
\date{20 Nov 2019}
\begin{document}

\maketitle
\begin{center}
    \includegraphics[width=60px]{Semantic_Web/images/semantic-web.pdf}
\end{center}
\section{Introduction}

\say{The Web was designed as an information space, with the goal that it should be useful not only for human-human communication, but also that machines would be able to participate and help. One of the major obstacles to this has been the fact that most information on the Web is designed for human consumption, and even if it was derived from a database with well defined meanings (in at least some terms) for its columns, that the structure of the data is not evident to a robot browsing the web. Leaving aside the artificial intelligence problem of training machines to behave like people, the Semantic Web approach instead develops languages for expressing information in a machine processable form.}

It is like a global database that enables several new operations, such as Search, Personalization, Linking and Integration.



\end{document}