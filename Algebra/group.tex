\documentclass{article}
\usepackage[utf8]{inputenc}
\usepackage{graphicx}
\usepackage{amsmath, amsthm, amssymb}
\usepackage{multicol}
\usepackage{svg}
\usepackage{caption}
\usepackage{vmargin}
\usepackage[hidelinks]{hyperref}
\usepackage[utf8]{inputenc}
\usepackage{epigraph}

\theoremstyle{definition}
\newtheorem{defi}{Definition}[subsection]
\newtheorem{theorem}{Theorem}[subsection]
\newtheorem{prop}{Proposition}[subsection]
\newtheorem{lemma}{Lemma}[subsection]

\title{Groups}
\author{Xiaozhe Yao\footnote{https://yaonotes.org/lecture-notes}}
\date{20 Nov 2019}
\begin{document}

\maketitle

\section{Definition and Properties}

\begin{defi}
\textbf{(Definition of Group)} let G be a set and $\star$ an "operation". It is a group if it satisfy the following properties:

1. \textbf{Closure:} $\forall (x,y) \in G \times G, x \star y \in G $

2. Operation is \textbf{associative} i.e $\forall (x, y, z) \in G \times G, (x \star y) \star z = x \star (y \star z) $

3. \textbf{Identity exists.} i.e $\exists e \in G,$ such that $\forall x \in G, e \star x = x $

4. \textbf{Inverse exists.} i.e $\forall x \in G, \exists y$ such that $x \star y = e$
\end{defi}

\begin{defi}
\textbf{(Definition of Abelian Group)} The group G is said to be commutative (or Abelian) if $\forall (x,y) \in G, x \star y = y \star x$

\noindent\textbf{Examples:} The integers $\mathbb{Z}$ form a group under the operation of addition, written as $(\mathbb{Z}, +)$. the identity (neutral element) is $0$, while the inverse is $-x$.
\end{defi}

\noindent It is often convenient to describe a group in terms of addition or multiplication table named \textbf{Cayley Table}. For example, the integers mod $n$ form a group under addition modulo $n$. Consider $\mathbb{Z}_5$, it consists the equivalence classes of the integers $0,1,2,3,4$. The operation on $\mathbb{Z}_5$ is modular addition. The element $0$ is the identity of the group and each element in $\mathbb{Z}_5$ has an inverse. For example, $2+3 = 3+2 = 0$. The following is the Cayley Table for $\mathbb{Z}_5$.

\begin{center}
\begin{tabular}{ |c|c|c|c|c|c| } 
\hline
+ & 0 & 1 & 2 & 3 & 4 \\
\hline
0 & 0 & 1 & 2 & 3 & 4 \\ 
1 & 1 & 2 & 3 & 4 & 0 \\ 
2 & 2 & 3 & 4 & 0 & 1 \\ 
3 & 3 & 4 & 0 & 1 & 2 \\ 
4 & 4 & 0 & 1 & 2 & 3 \\
\hline
\end{tabular}
\end{center}

\begin{defi}
\textbf{(Definition of Monoids)} A monoid is a set $(G, \star,$ associative, identity $e)$. If $\star$ is cumutative, $G$ is called Abelian. A submonoid is a subset $H$ of a monoid $G$ containing the identity $e$ and such that $\forall x, y \in H \Rightarrow xy \in H $.
\end{defi}

\begin{prop}
\textbf{(Total Commutativity)} If $G$ is an abelian monoid and $\phi$ is a bijection of $\{1,2,...,n\}$ onto itself then $\prod^{m}_{i=1}x_i$ = $\prod^{m}_{j=1}x_{\phi(j)}$.
\begin{proof}
By Induction. $\hfill \square$
\end{proof}
\end{prop}

\noindent\textbf{Example:}

1. ($\mathbb{N},+$) is a monoid.

2. Given a monoid $G$, the set ${\{x^{n}, n \in \mathbb{N} \}}$ is a submonoid, the proof is trivial.
\begin{defi}
\noindent\textbf{(Group Generator)} A subset $S \subset G$ is called a set of generators of G, denoted as $G=<S>$, if every element of $G$ can be expressed as a product of elements of $S$ or inverses of elements of $S$. In other words, any element $x \in G$ can be written as a finite product as 

$x = x_{1}^{n_{1}}...x_{m}^{n_{m}}$, $x_{i} \in S, n_{i} \in G, x \in G$

\noindent\textbf{Example:} The integers $\mathbb{Z}$ is an abelian additive group with identity 0. Since $\forall n \in \mathbb{Z}: n=1^{n}$, note that $1^{n}$ represents addition. $\mathbb{Z}=<1>$. Also $\mathbb{Z}=<-1>$.
\end{defi}
\begin{defi}
\noindent\textbf{(Cyclic Group)} A group is cyclic if it can be generated by a single element.

\noindent\textbf{Example:} $\mathbb{Z}$ is a cyclic group because it can be generated by $1$.
\end{defi}
\begin{prop}
Cyclic groups are all abelian, or commutative.
\begin{proof}
By looking into \textbf{Components Laws}. 
\end{proof}
\end{prop}

By definition, the cyclic group is generated by a single element, say $g$. Then we have $\forall a, b \in G, a=g^{m}, b=g^{n}$. then $ab = g^{m}g^{n} = g^{m+n} = g^{n+m} = g^ng^m = ba$.$\hfill \blacksquare$

\subsection{More Properties of Groups}

\begin{prop}
The identity element in a group $G$ is unique; there exists only one element $e \in G$ such that $eg = ge = g $ $\forall g \in G$.
\begin{proof}
Assume there are two identity, $e $ and $ e'$. Since $e$ is the identity, therefore $e\star e' = e'$. Meanwhile, since $e'$ is the identity, we have $e \star e' = e$. Therefore, $e = e'$.
\end{proof}
\end{prop}

\begin{prop}
For any element in Group g, the inverse of g, denoted by $g^{-1}$ is unique.
\begin{proof}
Assume there are two inverse for g, named $g‘$ and $g’‘$, we have $g \star g' = e$ and $g \star g'' = e$, then $g \star g' = g \star g''$, which means $g' = g''$.
\end{proof}
\end{prop}

\begin{prop}
Let G be a group, if $a, b \in G$ then $(ab)^{-1} = b^{-1}a^{-1}$.
\begin{proof}
$abb^{-1}a^{-1} = aea^{-1} = aa^{-1} = e $, Similiarly, $b^{-1}a^{-1}ab = e$, therefore $abb^{-1}a^{-1} = b^{-1}a^{-1}ab $, hence $(ab)^{-1} = b^{-1}a^{-1}$.
\end{proof}
\end{prop}

\begin{prop}
Let $G$ be a group, $\forall a \in G$, $(a^{-1})^{-1}=a$.
\begin{proof}
$a^{-1}(a^{-1})^{-1} = e$, multiply (operation) both sides by $a$, we then have: $(a^{-1})^{-1} = e(a^{-1})^{-1} = aa^{-1}(a^{-1})^{-1} = ae = a $.
\end{proof}
\end{prop}

\begin{prop}
Let $G$ be a group and $a, b$ be the only two elements in $G$. Then the equations $ax=b$ and $xa=b$ have a unique solution in G.
\begin{proof}
Suppose $ax=b$, then we must show that such an $x$ exists. We can multiply both sides of $ax=b$ by $a^{-1}$, then we have $a^{-1}ax = a^{-1}b$, i.e. $ex = a^{-1}b$, i.e. $x = a^{-1}b$.

To show uniqueness, suppose that $x_{1},x_{2}$ are both solutions of $ax=b$, then $ax_{1}=b=ax_{2}$. So $x_{1}=a^{-1}ax_{1} = a^{-1}ax_{2}=x_{2}$.

The proof for $xa=b$ is similar.
\end{proof}
\end{prop}

\begin{prop}
\noindent\textbf{(Right and Left Cancellation Laws)} If $G$ is a group and $a,b,c \in G$, then $ba=ca$ implies $b=c$ and $ab=ac$ implies $b=c$.
\begin{proof}
TBD.
\end{proof}
\end{prop}

\begin{prop}
\noindent\textbf{Proposition 2.2.7 (Components Laws):} Let $G$ be a group, then $\forall g,h \in G, \forall m,n \in \mathbb{Z}$, we have:

1. $g^{m}g^{n}=g^{m+n}$

2. $(g^{m})^{n} = g^{mn}$

3. $ (gh)^{n} = (h^{-1}g^{-1})^{-n} $ and if $G$ is abelian, then $(gh)^{n} = g^nh^n$.
\begin{proof}
TBD.
\end{proof}
\end{prop}

\section{Morphism}

\begin{defi}
\noindent\textbf{Definition of Morphism} Let $(G, \star)$ and $(F, \bullet)$ two groups. An application $f: G \to F$ is called a group morphism if $\forall (x, y) \in G$, $f(x \star y) = f(x) \bullet f(y)$
\end{defi}

\begin{prop}
We have the following properties:
\begin{itemize}
    \item If $e$ is the neutral element in $G$ and $e'$ is the neutral element in $F$, then $f(e)=e'$.
    \item $\forall x\in E$, $f(x^{-1})=(f(x))^{-1}$.
    \item $\forall x\in G, \forall n\in\mathbb{Z}$, $f(x^{n})=(f(x))^{n}$.
\end{itemize}
\end{prop}

\section{Product}
\begin{defi}
\textbf{Cartesian Product} Let $\{G_i\}_{i\in I}$ a family of groups $(G_i,\bullet)$. We consider $G=\times_{i\in I}G_i$ the Cartesian product, and we define the law: $\forall x,y \in G$, $x=(x_i)_{i\in I}$, $y=(y_i)_{i\in I}$, $x\bullet y=z,z = (z_i)_{i\in I}, z_i=x_i\bullet y_i$.

\begin{theorem}
$(G,\bullet)$ is a group with neutral element $0=(0_{G_i})_{i\in I}$.
\end{theorem}
\end{defi}
\section{Subgroups}

Subgroups are intuitively smaller groups inside a larger group. The operation must be \textbf{stable} for such smaller groups to be subgroups.

\subsection{Definitions}

\begin{defi}
\noindent\textbf{(Definition of Subgroups)} Let $(G, \star)$ be a group, for any group $(H, \star)$ such that $H \subseteq G$ and $H$ is still stable by the law $\star$, we call the group $(H, \star)$ a subgroup of group $G$. In this case, H is also thee group in its own right.

\end{defi}
Note that any group $G$ with at least two elements will always have at least two subgroups, which is either consisted of the neutral element alone or the entire group $G$. The subgroup $H={e}$ is called the \textit{trivial subgroup}. A subgroup that is a proper subset of $G$ is called a \textit{proper subgroup}.

\begin{prop}
\noindent\textbf{Proposition 3.1.2 (Subgroup Criteria):} A subset $H$ of $G$ is a subgroup if and only if it satisfies the following criteria:

1. The neutral element $e$ of $G$ is $\in H$.

2. If $h_1, h_2 \in H$, then $h_1h_2 \in H$. i.e. $H$ is stable by the operation.

3. If $h \in H$, then $h^{-1} \in H$

4. $H \neq \emptyset$ and $\forall(x,y)\in H^{2}$, $xy^{-1} \in H$.
\end{prop}

\subsection{Properties}

\begin{prop}
\label{fundamental_prop_subgroup}
Let $G$ be a group, $e\in G$ is the neutral, and $H\subset G$. The following properties are equivalent:
\begin{itemize}
    \item $H$ is a subgroup of $G$.
    \item $H$ is stable by $\cdot$, and $e\in H$, and $\forall x\in H$, $x^{-1}\in H$.
    \item $H$ is stable by $\cdot$, and $H \neq \emptyset$, and $\forall x\in H$, $x^{-1}\in H$.
    \item $H\neq \emptyset$, and $\forall (x,y)\in H^{2}, xy^{-1}\in H$.
\end{itemize}
\end{prop}

\begin{prop}
\noindent\textbf{Intersection of subgroups is still subgroup.} Let $G$ be a group, and $(H_{i})_{i \in I}$ is a family of subgroups of $G$. Then the intersection $H = \cap_{i \in I}H_{i}$ is a subgroup of $G$.
\begin{proof}
Since $\forall i \in I$, $e \in H_{i}$, $H=\cap_{i \in I}H_{i} \neq \emptyset$. In addition, $\forall x,y \in H$, $x,y \in H_{i}$ for some $i$. Since $H_{i}$ is a subgroup of $G$, by using \textbf{Proposition \ref{fundamental_prop_subgroup}}, we got $xy^{-1} \in H_{i}$. Hence $xy^{-1} \in \cap_{i \in I}H_{i}$. By using \textbf{\ref{fundamental_prop_subgroup}} again, we got $H$ is a subgroup of $G$.
\end{proof}
\end{prop}

\begin{prop}
Let $H$ be a subset of a group $G$. Then $H$ is a subgroup of $G$ if and only if $H\neq \emptyset$, and whenever $g,h \in H$ then $gh^{-1}$ is in $H$.
\begin{proof}
Assume $H$ is a subgroup of $G$. Since $h$ is in $H$m its inverse $h^{-1}$ must also be in $H$, then $gh^{-1}\in H$ due to the closure of the group operation.

Conversely, suppose $H\subset G$ such that $H\neq\emptyset$ and $gh^{-1}\in H$ whenever $g,h \in H$. If $g\in H$, then $gg^{-1}=e$ is in $H$. If $g\in H$, then $eg^{-1}=g^{-1}$ is also in $H$. Now let $h_1, h_2\in H$, we must show that their product is also in $H$. However, $h_1(h_2^{-1})^{-1}=h_1h_2\in H$. Hence $H $ is a subgroup of $G$.
\end{proof}
\end{prop}
\begin{defi}
\textbf{Generated Subgroups} We call subgroups of $G$ generated by $A\subset G$, and denote it by $gr(A)$, it indicates the intersection of all the groups of $G$ that contain $R$.

If $R$ is such that $gr(A) = G$, then we say $R$ generates $G$. By convention $gr(\emptyset)={e}$.
\end{defi}
\begin{theorem}
Let $G$ be a group with neutral $e$, and $A\subset G$. Then we have $H\subset G$, whose elements are either $e$ or something that can be written as $a_1^{\epsilon_1}a_2^{\epsilon_2}...a_n^{\epsilon_n}$ with $\epsilon_i\in {\pm1}$. Then $H$ is the subgroup of $G$ generated by $R$. The converse is also true.
\end{theorem}

\textit{Note: } If $H_1$ and $H_2$ are two subgroups of $G$, then $H_1\cup H_2$ is not a subgroup of $G$ in general. For example, in $G=\mathbb{Z}$ with the operation $+$. Let $H_1=\{x\in\mathbb{Z}, \exists p\in\mathbb{Z},x=2p\}$, $H_2=\{x\in\mathbb{Z}, \exists p\in\mathbb{Z},x=3p\}$. Then $H_1\cup H_2$ is not a subgroup of $\mathbb{Z}$ because $+$ is not stable anymore. $2\in H_1$, and $3\in H_2$ but $2+3$ is not in $\mathbb{Z}$, i.e. the addition is no longer stable.

\begin{theorem}
\textbf{Image of a Morphism, Inverse Image of a subgroup by a morphism}. Let $(G, \bullet)$ and $(F, \star)$ be groups, $f: G\to F$ a morphism, then
\begin{itemize}
    \item If $H$ is a subgroup of $G$, $f(H)$ is a subgroup of $F$.
    \item If $H\subset F$ is a subgroup of $F$, then $f^{-1}(H)=\{x\in G, \exists y\in H, y=f(x)\}$ is a subgroup of $G$.
\end{itemize}
\end{theorem}

There are two important examples: the image of $f$, $Im(f)={z\in F, \exists x\in G, z=f(x)}$ and the kernel of $f$, $ker(f)=f^{-1}(\{0_F\})=\{x\in G, f(x)=0_F\}$. We have the following important result:
\begin{theorem}
\textbf{Necessary and sufficient condition of injectivity of a morphism}. Let $f: G\to F$ a group morphism, then $f$ is injective if and only if $ker(f)=\{0_G\}$.
\end{theorem}

\begin{theorem}
\textbf{Nessary and sufficient condition of surjectivity of a morphism}. Let $f: G\to F$ a group morphism. $f$ is surjective if and only if $Im(f)=F$.
\end{theorem}

\begin{defi}
\textbf{Isomorphism} A morphism is bijective if and only $ker(f)=\{e_E\}$ and $Im(f)=F$. When the morphism is bijective, we say it is \textbf{isomorphism}. When $E=F$ an isomorphism is also called an automorphism.
\end{defi}

\section{Cyclic Groups}
Sometimes, a subgroup will solely depends on a single element of the group. By knowing the particular element will allow us computer any other element in the subgroup.
\begin{defi}
\textbf{Cyclic Group} If a group is generated by a single element, it is called a cyclic group. That is, it is a set of invertible elements with a single associative binary operation, and it contains an element $g$ such that every other element can be obtained. It can be written as $<g>$ and $g$ is called the generator of $G$.

\textit{Example: } Suppose that we consider $3\in\mathbb{Z}$ and look at all multiples (both positive and negative), it will be $3\mathbb{Z}={..., -3, 0, 3, 6}$. Every element in the subgroup is generated by ``3". 
\end{defi}

\begin{defi}
\textbf{Order} If $a$ is a generator of $G$, we define the order of $a$ to be the smallest positive integer $n$ such that $a^n=e$, and we write $|a|=n$. If there is no such integer n, we say that the order of $a$ is infinite and write $|a|=\infty$ to denote the order of $a$.
\end{defi}

\textit{Note:} Note that a cyclic group can have more than a single generator. Both $1$ and $5$ generate $\mathbb{Z}_6$.

\begin{theorem}
Every cyclic group is abelian.
\end{theorem}

\begin{theorem}
Every subgroup of a cyclic group is cyclic.
\end{theorem}

\begin{theorem}
Let $G$ be a cyclic group of order $n$ and suppose that $a$ is a generator for $G$. Then $a^{k}=e$ if and only if $n$ divides $k$.
\end{theorem}

\begin{theorem}
Let $G$ be a cyclic group of order $n$ and suppose that $a\in G$ is a generator of the group. If $b=a^{k}$, then the order of $b$ is $n/d$, where $d=gcd(k,n)$.
\end{theorem}

\begin{prop}
The generators of $\mathbb{Z}_n$ are the integers $r$ such that $1 \leq r <n$ and $gcd(r,n)=1$.
\end{prop}

\begin{theorem}
Let $G$ be a group and $a$ be any element in $G$. Then the set $<a>=\{a^{k}:k\in\mathbb{Z}\}$ is a subgroup of $G$. Furthermore, $<a>$ is the smallest subgrup of $G$ that contains $a$.
\textit{Note: } If we are using the ``+"" notation, as in the case of the integers under addition, we write it as $<a>=\{na: n\in\mathbb{Z}\}$.
\end{theorem}

\section{Cosets and Lagrange's Theorem}

\subsection{Cosets}
\begin{defi}
\textbf{Definition of Cosets} Let $G$ be a group and $H$ a subgroup of $G$. Define a \textbf{left coset} of $H$ with representative $g \in G$ to be the set $gH=\{gh: h \in H\}$. Similarly, we can define \textbf{right coset} by $Hg = \{hg: h\in H\}$. If there is no ambiguity, we will use the word \textit{coset} without specifying left or right.

\noindent\textbf{Example:} Let $H$ be the subgroup of $\mathbb{Z}_6$ consisting of the elements $0$ and $3$. The cosets are:
\end{defi}
1. $0+H=3+H=\{0,3\}$

2. $1+H=4+H=\{1,4\}$

3. $2+H=5+H=\{2,5\}$

\begin{prop}
In a commutative group, left and right cosets are always identical.
\end{prop}

\begin{lemma}
Let $H$ be a subgroup of a group $G$, and suppose that $g_1,g_2 \in G$. The following conditions are equivalent:

1. $g_1H=g_2H$

2. $Hg_{1}^{-1}=Hg_{2}^{-1}$

3. $g_1H \subset g_2H$

4. $g_2 \in g_1H$

5. $g_1^{-1}g_2 \in H$
\end{lemma}

\begin{theorem}
Let $H$ be a subgroup of a group $G$, then the left cosets of $H$ in $G$ is in partition $G$. That is, the group $G$ is the disjoint union of the left cosets of $H$ in $G$.
\textit{Note:} There is nothing special in this theorem about left cosets. Right cosets are also in partition $G$.
\end{theorem}

\begin{defi}
\textbf{Index} Let $G$ be a group and $H$ be a subgroup of $G$. Define the \textbf{index} of $H$ in $G$ to be the number of left cosets of $H$ in $G$, we can denote the index by $[G:H]$.
\end{defi}

\begin{theorem}
Let $H$ be a subgroup of a group $G$. The number of left cosets of $H$ in $G$ is the same as the number of right cosets of $H$ in $G$.
\end{theorem}

\subsection{Lagrange's Theorem}

\begin{prop}
Let $H$ be a subgroup of $G$ with $g\in G$ and define a map $\phi: H\to gH$ by $\phi(h)=gh$. The map $\phi$ is bijective; hence, the number of elements in $H$ is the same as the number of elements in $gH$.
\end{prop}

\begin{theorem}
\textbf{Lagrange} Let $G$ be a finite group and let $H$ be a subgroup of $G$. Then $|G|/|H| = [G:H]$ is the number of distinct left cosets of $H$ in $G$. In particular, the number of elements in $H$ must divide the number of elements in $G$. 
\textit{Note: } The converse of Lagrange's Theores is false.
\end{theorem}

\begin{prop}
Two cycles $\tau$ and $\mu$ in $S_n$ have the same length if and only if there exists a $\sigma\in S_n$ such that $\mu=\sigma\tau\sigma^{-1}$.
\end{prop}

\begin{prop}
Suppose that $G$ is a finite group and $g\in G$. Then the order of $g$ must divide the number of elements in $G$.
\end{prop}

\begin{prop}
Let $|G|=p$ with $p$ a prime number, then $G$ is cyclic and any $g\in G, g\neq e$ is a generator.

It suggests that groups of prime ordre $p$ must somehow look like $\mathbb{Z}_p$.
\end{prop}

\begin{prop}
Let $H$ and $K$ be subgroups of a finite group $G$ such that $K\subset H \subset G$, then $[G:K]=[G:H][H:K]$.
\end{prop}

\section{Distinguished Group}

Let $G$ be a group. For any $a\in G$, we can define an automorphism $\phi_a$ by $\phi_a(x)=axa^{-1}$. This is also called an interior automorphism. We have: 
\begin{itemize}
    \item $\phi_e=Id_{E}$.
    \item $\forall a,b \in G$, $\phi_a \circ \phi_b=\phi_{ab}$.
    \item $\phi_a\circ \phi_{a^{-1}} = Id_{G}: \phi_a$ is a bijection.
    \item $\forall x,y$ and $a\in G$, $\phi_a(x)\phi_a(y)=(axa^{-1})(aya^{-1})=axya^{-1}=\phi_a(xy)$: $\phi_a$ is an automorphism of $G$.
\end{itemize}

If we denote by $Aut(G)$ the set of automorphisms defined on $G$ with the law $\circ$, it is easy to see it is a group (not necessarily commutative, even if $G$ is commutative), and $\Psi: G\to Aut(G), a\to \phi_a$ is a morphism between $G$ and $Aut(G)$. We can look for its kernel as $Ker(\Psi)=\{a \in G, \forall x\in G, ax=xa\}$. It is called the center of $G$, denoted by $Z(G)$: the set of elements that communte with all elements of $G$. When a group is commutative, $Z(G)=G$. If $H$ is a subgroup of $G$, and $a\in G$, $\phi_a(H)$ is a subgroup of $G$, denoted by $aHa^{-1}$. 

\begin{defi}
\textbf{Distinguished Groups} A subgroup $H$ of $G$ is called distinguished if it is stable by any automorphism $\phi_a: \forall a\in G, aHa^{-1}=H$, i.e. $\forall x\in H, axa^{-1}\in H$.
\end{defi}

\begin{prop}
If $H$ is a distinguished subgroup of $G$, it is stable and invariant by any interior automorphism.
\begin{proof}
Let $H$ be a distinguished subgroup of $H$, for any $a\in G$, we already have $aHa^{-1}\subset H$. We need to show that $H\subset aHa^{-1}$.
Let any $(a,h)\in G\times H$. Since $\phi_a\circ \phi_{a^{-1}}=Id_{G}$, we see that $h$ is the image by $\phi_a$ of $\phi_{a^{-1}}\in H$ since $H$ is stable by $\phi_{a^{-1}}$. This shows that $H\subset aHa^{-1}$, and hence $aHa^{-1}=H$.
\end{proof}
\end{prop}

\begin{prop}
We have the following properties:
\begin{itemize}
    \item $\{e\}$ is a distinguished subgroup.
    \item Any subgroup of an abelian group is distinguished.
\end{itemize}
\end{prop}

\begin{theorem}
Let $f: G \to F$ be a group morphism. If $H$ is a distinguished subgroup of $G$, $f(H)$ is a distinguished subgroup of $f(G)$. If $H'$ is a distinguished subgroup of $F$, $f^{-1}(H')$ is a distinguished subgroup of $G$.
\end{theorem}

\begin{prop}
$Ker(f)$ is distinguished. If one wants to prove that $H\subset G$ is distinguished, one way of doing so is to identify: $f: G\to F$ for a suitable $F$ such that $H=Ker(f)$.
\end{prop}

\section{Quotient Group}

\begin{defi}
\textbf{Compatible Equivalent relations} If $G$ is a group and $\mathcal{R}$ an equivalent relation, we say it is compatible on the left with the composition law $(x\cdot y)$ if for any $z\in G$ and any $x,y$ such that $x\mathcal{R}y$ then $zx\mathcal{R}zy$. It is compatible on the right if $xz\mathcal{R}yz$.
\end{defi}

\begin{theorem}
Let $G$ be a group, any equivalent relations that are compatible on the left (resp. on the right) with the composition law in $G$ is of the form $x^{-1}y\in H$ (resp. $yx^{-1}\in H$) where $H$ is a subgroup of $G$. Conversely, any relation of this type is an equivalent relation that are compatible with the composition in $G$.

Conversely, define $\mathcal{H}$ the relation $x\mathcal{H}y$ if and only if $x^{-1}y\in H$, and we assume that $H$ is a group. This is an equivalent relation:
\begin{itemize}
    \item It is reflexive: $x^{-1}x=e\in H$ so $x\mathcal{H}x$.
    \item It is symmetric: $x\mathcal{H}y \implies x^{-1}y\in H \implies y^{-1}x = (x^{-1}y)^{-1}\in H \implies y\mathcal{H}x$.
    \item It is transtive: $x\mathcal{H}y and y\mathcal{H}z \implies x^{-1}y\in H and y^{-1}z \in H \implies (x^{-1}y)(y^{-1}z)=x^{-1}z\in H$, as $H$ is stable.
    \end{itemize}
\end{theorem}

\begin{defi}
\textbf{Left and Right classes} Let $G$ be a group, $H$ a subgroup, $a\in G$. Then the set $aH=\{ah| h\in H\}$ is called left class of $a$, and the set $Ha=\{ha|h\in H\}$ is called right class of $a$, according to $H$. These are the equivalent classes for the relations $\mathcal{R}_l$ and $\mathcal{R}_r$ defined by: $\forall (x,y)\in G^{2}, x\mathcal{R}_l y \Longleftrightarrow x^{-1}y\in H$ and $\forall (x,y)\in G^{2}, x\mathcal{R}_r y \Longleftrightarrow yx^{-1}\in H$.
\end{defi}

\begin{prop}
Let $G$ be a finite group, and $H$ a subgroup of G. then $H$ is a finite group and $|H|$ divides $|G|$.
\end{prop}

Assume we have an equivalent relation, we have the set of equivalent classes $G/\mathcal{R}$, we have a natural injection: $i: G\to G/\mathcal{R}, x\to \overline{x}$.

\begin{theorem}
Let $G$ be a group, the equivalent relations that are compatible with the group structure of $G$ are those of the form $x^{-1}y\in H$ where $H$ is a distinguished subgroup. The equivalence is compatible with the group strcture of $G$ means that $x\mathcal{R}y$ and $x'\mathcal{R}y'\implies (xx')\mathcal{R}(yy')$.
\end{theorem}

\begin{theorem}
Let $G$ be a group, $H$ a distinguished subgroup, and let us consider $G/H$, the set of equivalent classes defined by the relation $x^{-1}y\in H$. Then the law $\overline{x}\cdot\overline{y}=\overline{xy}$ defines a group structure on $G/H$. The canonical injection $G\to G/H$ is a group morphism. If in addition, $G$ is an abelian group, $G/H$ is also an abelian group.
\end{theorem}

\begin{theorem}
Let $G$ and $F$ be two groups, $f: G\to F$ a group morphism, let $\phi$ the canonical surjection $G\to G/ker(f)$, $j$ the canonical injection from $G/ker(f)$ to $F$. Then there exists a unique group morphism $\overline{f}$ from $G/ker(f)$ to $Im(f)$ such that $f=j\circ \overline{f}\circ \phi$ and it is an isomorphism.
\end{theorem}

\begin{theorem}
If $H$ is a subgroup of $(\mathbb{Z},+)$, then there exists $p\in\mathbb{Z}$ such that $H=p\mathbb{Z}$.
\end{theorem}

\begin{defi}
$\forall n\in\mathbb{N}$, the group $\mathbb{Z}/n\mathbb{Z}$ is called the group of integer number module $n$. If $n=0$, it is then isomorphic to $\mathbb{Z}$. If $n\neq 0$, it is a cyclic group, i.e. there exists $m\in\mathbb{N}$ such that for any $x\in \mathbb{Z}/n\mathbb{Z}$, $mx=0$: indeed $nx=0$.
\end{defi}

\section{Symmetric Group}
\begin{defi}
\textbf{Permutations} Let $E$ be a set, we denote $S_E$ tje set of bijection from $E$ onto $F$, they are called permutations.
\end{defi}
\begin{theorem}
$(S_E,\circ)$ is a group, called the symmetric group.
\end{theorem}
\begin{theorem}
\textbf{Cayley} Any finite group is isomorphic to a subgroup of $S_|E|$.
\end{theorem}
\begin{defi}
\textbf{Orbits, Cycle, Transposition} Let $s\in S_n$, and we define the relation $xR_sy$ if and only if there exists $p\in\mathbb{N}$ such that $y=s^{p}(x)$. The equivalent classes are called orbits.

A cycle is a permutation such that there exists only one orbit that is not reduced to a point. This orbit is called the support of the cycle, and its cardinal is the length of the permutation. We call transposition a cycle of length $2$. If $\sigma$ is a cycle, the set of points such that $\sigma(a)\neq a$ is called the support of $\sigma$.
\end{defi}

\begin{theorem}
Any permutation of $\sigma\in S_n$ is the product of cycles of disjoint support. 
\end{theorem}

\begin{theorem}
Any transposition is a product of transpositions.
\end{theorem}

\begin{theorem}
We have the following corollary:
\begin{itemize}
    \item if $\sigma$ is the product of $h$ transposition, then $\epsilon(\sigma)=(-1)^{h}$.
    \item $\epsilon$ is a morphism between $S_n$ and $\{+1,-1\}$.
\end{itemize}
\end{theorem}

\begin{defi}
The kernel of $\epsilon$ is a distinguished group called the alternated group $A_n$. Its element have signature $+1$ are called the even permutations, the elements of $S_n-A_n$ have signature $-1$ and are called odd permutations.
\end{defi}
\end{document}