\documentclass{article}
\usepackage[utf8]{inputenc}
\usepackage{graphicx}
\usepackage{amsmath, amsthm, amssymb}
\usepackage{multicol}
\usepackage{svg}
\usepackage{caption}
\usepackage{vmargin}
\usepackage[hidelinks]{hyperref}
\usepackage[utf8]{inputenc}
\usepackage{epigraph}

\theoremstyle{definition}
\newtheorem{defi}{Definition}[subsection]
\newtheorem{theorem}{Theorem}[subsection]
\newtheorem{prop}{Proposition}[subsection]
\newtheorem{lemma}{Lemma}[subsection]

\title{Groups}
\author{Xiaozhe Yao\footnote{https://yaonotes.org/lecture-notes}}
\date{20 Nov 2019}
\begin{document}

\maketitle

\section{Definition and Properties}

\begin{defi}
\textbf{(Definition of Group)} let G be a set and $\star$ an "operation". It is a group if it satisfy the following properties:

1. \textbf{Closure:} $\forall (x,y) \in G \times G, x \star y \in G $

2. Operation is \textbf{associative} i.e $\forall (x, y, z) \in G \times G, (x \star y) \star z = x \star (y \star z) $

3. \textbf{Identity exists.} i.e $\exists e \in G,$ such that $\forall x \in G, e \star x = x $

4. \textbf{Inverse exists.} i.e $\forall x \in G, \exists y$ such that $x \star y = e$
\end{defi}

\begin{defi}
\textbf{(Definition of Abelian Group)} The group G is said to be commutative (or Abelian) if $\forall (x,y) \in G, x \star y = y \star x$

\noindent\textbf{Examples:} The integers $\mathbb{Z}$ form a group under the operation of addition, written as $(\mathbb{Z}, +)$. the identity (neutral element) is $0$, while the inverse is $-x$.
\end{defi}

\noindent It is often convenient to describe a group in terms of addition or multiplication table named \textbf{Cayley Table}. For example, the integers mod $n$ form a group under addition modulo $n$. Consider $\mathbb{Z}_5$, it consists the equivalence classes of the integers $0,1,2,3,4$. The operation on $\mathbb{Z}_5$ is modular addition. The element $0$ is the identity of the group and each element in $\mathbb{Z}_5$ has an inverse. For example, $2+3 = 3+2 = 0$. The following is the Cayley Table for $\mathbb{Z}_5$.

\begin{center}
\begin{tabular}{ |c|c|c|c|c|c| } 
\hline
+ & 0 & 1 & 2 & 3 & 4 \\
\hline
0 & 0 & 1 & 2 & 3 & 4 \\ 
1 & 1 & 2 & 3 & 4 & 0 \\ 
2 & 2 & 3 & 4 & 0 & 1 \\ 
3 & 3 & 4 & 0 & 1 & 2 \\ 
4 & 4 & 0 & 1 & 2 & 3 \\
\hline
\end{tabular}
\end{center}

\begin{defi}
\textbf{(Definition of Monoids)} A monoid is a set $(G, \star,$ associative, identity $e)$. If $\star$ is cumutative, $G$ is called Abelian. A submonoid is a subset $H$ of a monoid $G$ containing the identity $e$ and such that $\forall x, y \in H \Rightarrow xy \in H $.
\end{defi}

\begin{prop}
\textbf{(Total Commutativity)} If $G$ is an abelian monoid and $\phi$ is a bijection of $\{1,2,...,n\}$ onto itself then $\prod^{m}_{i=1}x_i$ = $\prod^{m}_{j=1}x_{\phi(j)}$.
\end{prop}
\textit{Proof:} By Induction. $\hfill \square$

\noindent\textbf{Example:}

1. ($\mathbb{N},+$) is a monoid.

2. Given a monoid $G$, the set ${\{x^{n}, n \in \mathbb{N} \}}$ is a submonoid, the proof is trivial.
\begin{defi}
\noindent\textbf{(Group Generator)} A subset $S \subset G$ is called a set of generators of G, denoted as $G=<S>$, if every element of $G$ can be expressed as a product of elements of $S$ or inverses of elements of $S$. In other words, any element $x \in G$ can be written as a finite product as 

$x = x_{1}^{n_{1}}...x_{m}^{n_{m}}$, $x_{i} \in S, n_{i} \in G, x \in G$

\noindent\textbf{Example:} The integers $\mathbb{Z}$ is an abelian additive group with identity 0. Since $\forall n \in \mathbb{Z}: n=1^{n}$, note that $1^{n}$ represents addition. $\mathbb{Z}=<1>$. Also $\mathbb{Z}=<-1>$.
\end{defi}
\begin{defi}
\noindent\textbf{(Cyclic Group)} A group is cyclic if it can be generated by a single element.

\noindent\textbf{Example:} $\mathbb{Z}$ is a cyclic group because it can be generated by $1$.
\end{defi}
\begin{prop}
Cyclic groups are all abelian, or commutative.
\end{prop}
\textit{Proof:} By looking into \textbf{Components Laws}. 

By definition, the cyclic group is generated by a single element, say $g$. Then we have $\forall a, b \in G, a=g^{m}, b=g^{n}$. then $ab = g^{m}g^{n} = g^{m+n} = g^{n+m} = g^ng^m = ba$.$\hfill \blacksquare$

\section{Basic Properties of Groups}

\begin{prop}
The identity element in a group $G$ is unique; there exists only one element $e \in G$ such that $eg = ge = g $ $\forall g \in G$.
\end{prop}
\textit{Proof:} Assume there are two identity, $e $ and $ e'$. Since $e$ is the identity, therefore $e\star e' = e'$. Meanwhile, since $e'$ is the identity, we have $e \star e' = e$. Therefore, $e = e'$. $\hfill\blacksquare$

\begin{prop}
For any element in Group g, the inverse of g, denoted by $g^{-1}$ is unique.
\end{prop}
\textit{Proof:} Assume there are two inverse for g, named $g‘$ and $g’‘$, we have $g \star g' = e$ and $g \star g'' = e$, then $g \star g' = g \star g''$, which means $g' = g''$. $\hfill\blacksquare$

\begin{prop}
Let G be a group, if $a, b \in G$ then $(ab)^{-1} = b^{-1}a^{-1}$
\end{prop}
\textit{Proof:} $abb^{-1}a^{-1} = aea^{-1} = aa^{-1} = e $, Similiarly, $b^{-1}a^{-1}ab = e$, therefore $abb^{-1}a^{-1} = b^{-1}a^{-1}ab $, hence $(ab)^{-1} = b^{-1}a^{-1}$. $\hfill\blacksquare$

\begin{prop}
Let $G$ be a group, $\forall a \in G$, $(a^{-1})^{-1}=a$
\end{prop}
\textit{Proof:} $a^{-1}(a^{-1})^{-1} = e$, multiply (operation) both sides by $a$, we then have: $(a^{-1})^{-1} = e(a^{-1})^{-1} = aa^{-1}(a^{-1})^{-1} = ae = a $. $\hfill\blacksquare$

\begin{prop}
Let $G$ be a group and $a, b$ be the only two elements in $G$. Then the equations $ax=b$ and $xa=b$ have a unique solution in G.
\end{prop}
\textit{Proof:} Suppose $ax=b$, then we must show that such an $x$ exists. We can multiply both sides of $ax=b$ by $a^{-1}$, then we have $a^{-1}ax = a^{-1}b$, i.e. $ex = a^{-1}b$, i.e. $x = a^{-1}b$.

To show uniqueness, suppose that $x_{1},x_{2}$ are both solutions of $ax=b$, then $ax_{1}=b=ax_{2}$. So $x_{1}=a^{-1}ax_{1} = a^{-1}ax_{2}=x_{2}$.

The proof for $xa=b$ is similiar. $\hfill\blacksquare$

\begin{prop}
\noindent\textbf{(Right and Left Cancellation Laws)} If $G$ is a group and $a,b,c \in G$, then $ba=ca$ implies $b=c$ and $ab=ac$ implies $b=c$.
\end{prop}

\textit{Proof:} $\hfill\blacksquare$

\begin{prop}
\noindent\textbf{Proposition 2.2.7 (Components Laws):} Let $G$ be a group, then $\forall g,h \in G, \forall m,n \in \mathbb{Z}$, we have:

1. $g^{m}g^{n}=g^{m+n}$

2. $(g^{m})^{n} = g^{mn}$

3. $ (gh)^{n} = (h^{-1}g^{-1})^{-n} $ and if $G$ is abelian, then $(gh)^{n} = g^nh^n$.
\end{prop}

\textit{Proof:} $\hfill\blacksquare$

\noindent\textbf{}

\section{Subgroups}

Subgroups are intuitively smaller groups inside a larger group. The operation must be \textbf{stable} for such smaller groups to be subgroups.

\section{Definitions}

\begin{defi}
\noindent\textbf{(Definition of Subgroups)} Let $(G, \star)$ be a group, for any group $(H, \star)$ such that $H \subseteq G$ and $H$ is still stable by the law $\star$, we call the group $(H, \star)$ a subgroup of group $G$. In this case, H is also thee group in its own right.

\end{defi}
Note that any group $G$ with at least two elements will always have at least two subgroups, which is either consisted of the neutral element alone or the entire group $G$. The subgroup $H={e}$ is called the \textit{trivial subgroup}. A subgroup that is a proper subset of $G$ is called a \textit{proper subgroup}.

\begin{prop}
\noindent\textbf{Proposition 3.1.2 (Subgroup Criteria):} A subset $H$ of $G$ is a subgroup if and only if it satisfies the following criteria:

1. The neutral element $e$ of $G$ is $\in H$.

2. If $h_1, h_2 \in H$, then $h_1h_2 \in H$. i.e. $H$ is stable by the operation.

3. If $h \in H$, then $h^{-1} \in H$

4. $H \neq \emptyset$ and $\forall(x,y)\in H^{2}$, $xy^{-1} \in H$.
\end{prop}

\section{Properties}

\begin{theorem}
\noindent\textbf{(Intersection of Groups)} Let $G$ be a group, and $(H_{i})_{i \in I}$ is a family of subgroups of $G$. Then the intersection $H = \cap_{i \in I}H_{i}$ is a subgroup of $G$.
\end{theorem}
\textit{Proof:} Since $\forall i \in I$, $e \in H_{i}$, $H=\cap_{i \in I}H_{i} \neq \emptyset$. In addition, $\forall x,y \in H$, $x,y \in H_{i}$ for some $i$. Since $H_{i}$ is a subgroup of $G$, by using \textbf{Proposition 3.1.2 (4)}, we got $xy^{-1} \in H_{i}$. Hence $xy^{-1} \in \cap_{i \in I}H_{i}$. By using \textbf{Proposition 3.1.2 (4)} again, we got $H$ is a subgroup of $G$. $\hfill\blacksquare$

\begin{defi}
\noindent\textbf{Definition 3.2.2} We call subgroups of G generated by $A \subset G$, and denote it by $gr(A)$
\end{defi}


\noindent\textbf{Theorem 3.2.3}

\section{Cosets and Lagrange's Theorem}

\begin{defi}
\textbf{Definition of Cosets} Let $G$ be a group and $H$ a subgroup of $G$. Define a \textbf{left coset} of $H$ with representative $g \in G$ to be the set $gH=\{gh: h \in H\}$. Similarly, we can define \textbf{right coset} by $Hg = \{hg: h\in H\}$. If there is no ambiguity, we will use the word \textit{coset} without specifying left or right.

\noindent\textbf{Example:} Let $H$ be the subgroup of $\mathbb{Z}_6$ consisting of the elements $0$ and $3$. The cosets are:
\end{defi}
1. $0+H=3+H=\{0,3\}$

2. $1+H=4+H=\{1,4\}$

3. $2+H=5+H=\{2,5\}$

\begin{prop}
In a commutative group, left and right cosets are always identical.
\end{prop}

\begin{lemma}
Let $H$ bee a subgroup of a group $G$, and suppose that $g_1,g_2 \in G$. The following conditions are equivalent:

1. $g_1H=g_2H$

2. $Hg_{1}^{-1}=Hg_{2}^{-1}$

3. $g_1H \subset g_2H$

4. $g_2 \in g_1H$

5. $g_1^{-1}g_2 \in H$
\end{lemma}

\begin{theorem}
Let 1
\end{theorem}

\section{Morphism and Product}

\begin{defi}
\noindent\textbf{(Definition of Morphism)} Let $(G, \star)$ and $(F, \bullet)$ two groups. A projection $f: G \to F$ is called a group morphism if $\forall (x, y) \in G$, $f(x \star y) = f(x) \bullet f(y)$

For example, we have $(G, \star)$ and $((\mathbb{Z}, +)$, we can then define

$f: \mathbb{Z} \to \mathbb{Z}$
\end{defi}


\section{Other Properties of Group}

\section{Distinguished Group}

\section{Quotient Group}

\section{Symmetric Group}

\section{Summary}

\end{document}