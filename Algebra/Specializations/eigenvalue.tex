\documentclass{article}
\usepackage[utf8]{inputenc}
\usepackage{graphicx}
\usepackage{amsmath, amsthm, amssymb}
\usepackage{multicol}
\usepackage{svg}
\usepackage{caption}
\usepackage{vmargin}
\usepackage[hidelinks]{hyperref}
\title{Eigen Value, Eigen Vectors}
\author{Xiaozhe Yao}
\date{20 Feb 2020}
\begin{document}

\maketitle
\section{Definition}
In linear algebra, an eigen vectore, or characteristic vector of a linear transformation $\mathbb{A}$, is a non-zero vector that changes at most by a scalar factor when that linear transformation is applied to it. i.e. $\mathbb{A}v=\lambda v$ where $\lambda$ is a scalar, and is called eigen value.
\end{document}