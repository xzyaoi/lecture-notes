\documentclass{article}
\usepackage[utf8]{inputenc}
\usepackage{graphicx}
\usepackage{amsmath, amsthm, amssymb}
\usepackage{multicol}
\usepackage{svg}
\usepackage{caption}
\usepackage{vmargin}
\usepackage[hidelinks]{hyperref}
\usepackage[utf8]{inputenc}
\usepackage{epigraph}

\theoremstyle{definition}
\newtheorem{defi}{Definition}[subsection]
\newtheorem{theorem}{Theorem}[subsection]
\newtheorem{prop}{Proposition}[subsection]
\newtheorem{lemma}{Lemma}[subsection]

\title{Ring}
\author{Xiaozhe Yao\footnote{https://yaonotes.org/lecture-notes}}
\date{20 Nov 2019}
\begin{document}

\maketitle

\section{Definition}
\begin{defi}
We call ring any set $R$, equipped with two operations, $+$ and $\cdot$ such that: \begin{itemize}
    \item $(R,+)$ is a commutative group, its neutral is called $0$.
    \item The internal law $\cdot$ is associative and has a neutral element denoted by $1$.
    \item The multiplication is distributive with respect to the addition: $\forall (a,b,c)\in \mathbb{R}$, $a\cdot (b+c)=a\cdot b + a \cdot c$.
\end{itemize}
A ring is said to be commutative if $\cdot$ is commutative.
\end{defi}

\begin{theorem}
Let $R$ be a ring, then we have the following theorems:\begin{itemize}
    \item $\forall x\in R, 0\times x = x \times 0 = 0$.
    \item $\forall (x,y)\in R^{2}, x\times (-y) = (-x)\times y = -x \times y$.
    \item $\forall x\in R, \forall n\in \mathbb{Z}$, we have $x\cdot(ny)=(nx)\cdot y = n(x\cdot y)$.
    \item $\forall (x,y)\in R^{2}, (-x\cdot -y) = x \cdot y$.
\end{itemize}
\end{theorem}
\begin{defi}
\textbf{Integer Ring} We say that a ring is integer if $R$ is not reduced to ${0}$, is commutative, and for $\forall (a,b)\in R^{2}, ab=0\implies (a=0)$ or $(b=0)$.

If $a\in R$, $R$ is a ring that is not reduced to 0, is such that $\forall b\in R$, $ab=0$. We say that $a$ is a divisor of 0.
\end{defi}

\begin{defi}
\textbf{Morphisms of Ring} Let $R$ and $R'$ be two rings. An application $f: A\to R'$ is said to be a morphism of rings if the following conditions hold true:
\begin{itemize}
    \item $\forall (x,y)\in R^{2}, f(x+y)=f(x)+f(y)$: $f$ is a morphism of the group $(R,+)$.
    \item $\forall(x,y)\in R^{2}, f(xy)=f(x)f(y)$.
    \item $f(1^{R})=1^{R'}$.
\end{itemize}
\end{defi}
\end{document}