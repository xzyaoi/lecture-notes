\documentclass[12pt,openany]{book}
\setlength{\headheight}{15pt}
\usepackage{amsmath, amsthm, amssymb}
\usepackage{mdframed}
\usepackage{lipsum}

\newmdtheoremenv{thm}{Theorem}[section]
\newmdtheoremenv{exmp}{Example}[section]
\newtheorem*{tips}{Tips}

\theoremstyle{definition}
\newtheorem{defi}{Definition}[section]
\newtheorem*{lemma}{Lemma}
\newtheorem{cor}{Corollary}[section]
\newenvironment{soln}{\begin{proof}[Solution]}{\end{proof}}
\newenvironment{comment}{\begin{proof}[Comment]}{\end{proof}}
\newenvironment{motivation}{\begin{proof}[Motivation]}{\end{proof}}
\newtheorem{psol}{Problem}[section]
\newtheorem{prob}{Problem}[section]
\newtheorem{hint}{Hint}[section]
\usepackage{amsthm,amssymb,amsmath}
\theoremstyle{definition}
\setcounter{section}{1}
\newtheorem*{case}{Example}
\newtheorem{tip}{Tip}[section]
\usepackage{titlesec}
\titleformat{\chapter}[display]
{\normalfont\bfseries\filcenter}
{\LARGE\thechapter}
{1ex}
{\titlerule[2pt]
\vspace{2ex}%
\LARGE}
[\vspace{1ex}%
{\titlerule[2pt]}]


\usepackage{cancel}
\usepackage[margin=4cm]{geometry}
\usepackage{hyperref}
\usepackage{fancyhdr}
\pagestyle{fancy}
\fancyhead{}
\fancyfoot{}
\lhead{Group Theory}
\chead{Xiaozhe Yao}
\rhead{\thepage}
\newenvironment{dedication}
    {\vspace{6ex}\begin{quotation}\begin{center}\begin{em}}
    {\par\end{em}\end{center}\end{quotation}}
    
\newcommand{\HRule}{\rule{\linewidth}{0.5mm}} % Defines a new command for the horizontal lines, change thickness here
\title{Algebra}
\begin{document}
% Center everything on the page
 

%----------------------------------------------------------------------------------------
%	TITLE SECTION
%----------------------------------------------------------------------------------------
\begin{center}
\HRule \\[0.4cm]
{ \huge \bfseries Group Theory}\\[0.4cm] % Title of your document
\HRule \\[1.5cm]
\begin{minipage}{0.4\textwidth}
\begin{flushleft} \large
\emph{Authors}\\
Xiaozhe Yao \newline
www.yaonotes.org \newline

\end{flushleft}
\end{minipage}
~
\begin{minipage}{0.4\textwidth}
\begin{flushright} \large

\end{flushright}
\end{minipage}\\[0.5cm]
\end{center}

\chapter{Introduction}

\chapter{Definition}

\section{Definition and Properties}

\textbf{Definition 2.1 (Definition of Group):} let G be a set and $\star$ an "operation", i.e. an application defined from G $\times$ D onto G:

$\forall (x,y) \in G \times G, x \star y \in G $

that has the following properties:





\end{document}