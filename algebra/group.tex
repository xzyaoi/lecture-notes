\documentclass[12pt,openany]{book}
\setlength{\headheight}{15pt}
\usepackage{amsmath, amsthm, amssymb}
\usepackage{mdframed}
\usepackage{lipsum}

\newmdtheoremenv{thm}{Theorem}[section]
\newmdtheoremenv{exmp}{Example}[section]
\newtheorem*{tips}{Tips}

\theoremstyle{definition}
\newtheorem{defi}{Definition}[section]
\newtheorem*{lemma}{Lemma}
\newtheorem{cor}{Corollary}[section]
\newenvironment{soln}{\begin{proof}[Solution]}{\end{proof}}
\newenvironment{comment}{\begin{proof}[Comment]}{\end{proof}}
\newenvironment{motivation}{\begin{proof}[Motivation]}{\end{proof}}
\newtheorem{psol}{Problem}[section]
\newtheorem{prob}{Problem}[section]
\newtheorem{hint}{Hint}[section]
\usepackage{amsthm,amssymb,amsmath}
\theoremstyle{definition}
\setcounter{section}{1}
\newtheorem*{case}{Example}
\newtheorem{tip}{Tip}[section]
\usepackage{titlesec}
\titleformat{\chapter}[display]
{\normalfont\bfseries\filcenter}
{\LARGE\thechapter}
{1ex}
{\titlerule[2pt]
\vspace{2ex}%
\LARGE}
[\vspace{1ex}%
{\titlerule[2pt]}]


\usepackage{cancel}
\usepackage[margin=4cm]{geometry}
\usepackage{hyperref}
\usepackage{fancyhdr}
\pagestyle{fancy}
\fancyhead{}
\fancyfoot{}
\lhead{Group Theory}
\chead{Xiaozhe Yao}
\rhead{\thepage}
\newenvironment{dedication}
    {\vspace{6ex}\begin{quotation}\begin{center}\begin{em}}
    {\par\end{em}\end{center}\end{quotation}}
    
\newcommand{\HRule}{\rule{\linewidth}{0.5mm}} % Defines a new command for the horizontal lines, change thickness here
\title{Algebra}
\begin{document}
% Center everything on the page
 

%----------------------------------------------------------------------------------------
%	TITLE SECTION
%----------------------------------------------------------------------------------------
\begin{center}
\HRule \\[0.4cm]
{ \huge \bfseries Group Theory}\\[0.4cm] % Title of your document
\HRule \\[1.5cm]
\begin{minipage}{0.4\textwidth}
\begin{flushleft} \large
\emph{Authors}\\
Xiaozhe Yao \newline
www.yaonotes.org \newline

\end{flushleft}
\end{minipage}
~
\begin{minipage}{0.4\textwidth}
\begin{flushright} \large

\end{flushright}
\end{minipage}\\[0.5cm]
\end{center}

\chapter{Introduction}

Group is central to algebra, other structures, such as rings, fields and vector spaces can be seen as a group with additional operations and axioms. Group play an important role in physics, computer science, etc. It is also central to public key cryptography.

\chapter{Definition}

\section{Definition and Properties}

\noindent\textbf{Definition 2.1 (Definition of Group):} let G be a set and $\star$ an "operation". It is a group if it satisfy the following properties:

1. \textbf{Closure:} $\forall (x,y) \in G \times G, x \star y \in G $

2. Operation is \textbf{associative} i.e $\forall (x, y, z) \in G \times G, (x \star y) \star z = x \star (y \star z) $

3. \textbf{Identity exists.} i.e $\exists e \in G,$ such that $\forall x \in G, e \star x = x $

4. \textbf{Inverse exists.} i.e $\forall x \in G, \exists y$ such that $x \star y = e$

\noindent\textbf{Definition 2.2 (Definition of Abelian Group):} The group G is said to be commutative (or Abelian) if 

$\forall (x,y) \in G, x \star y = y \star x$

\noindent\textbf{Examples:} The integers \mathbb{Z} form a group under the operation of addition, written as $(\mathbb{Z}, +)$. the identity (neutral element) is $0$, while the inverse is $-x$.

\noindent It is often convenient to describe a group in terms of adadition or multiplication table named \textbf{Cayley Table}. For example, the integers mood $n$ form a group under addition modulo $n$. Consider $\mathbb{Z}_5$, it consists the equivalence classes of the integers $0,1,2,3,4$. The operation on $\mathbb{Z}_5$ is modular addition. The element $0$ is the identity of the group and each element in $\mathbb{Z}_5$ has an inverse. For example, $2+3 = 3+2 = 0$. The following is the Cayley Table for $\mathbb{Z}_5$.

\begin{center}
\begin{tabular}{ |c|c|c|c|c|c| } 
\hline
+ & 0 & 1 & 2 & 3 & 4 \\
\hline
0 & 0 & 1 & 2 & 3 & 4 \\ 
1 & 1 & 2 & 3 & 4 & 0 \\ 
2 & 2 & 3 & 4 & 0 & 1 \\ 
3 & 3 & 4 & 0 & 1 & 2 \\ 
4 & 4 & 0 & 1 & 2 & 3 \\
\hline
\end{tabular}
\end{center}

\section{Basic Properties of Groups}

\textbf{Proposition 2.2.1} The identity element in a group $G$ is unique; there exists only one element $e \in G$ such that $eg = ge = g $ $\forall g \in G$.

\textit{Proof:} Assume there are two identity, $e $ and $ e'$. Since $e$ is the identity, therefore $e\stare'$

 \chap te = r{Morphism and Product}

\noindent\textbf{Definition 3.1 (Definition of Morphism):} Let $(G, \star)$ and $(F, \bullet)$ two groups. A projection $f: G \to F$ is called a grooup morphism if $\forall (x, y) \in G$, $f(x \star y) = f(x) \bullet f(y)$

For example, we have $(G, \star)$ and $((\mathbb{Z}, +)$, we can then define

$f: \mathbb{Z} \to \mathbb{Z}$

\chapter{Subgroups}

\chapter{Other Properties of Group}

\chapter{Distinguished Group}

\chapter{Quotient Group}

\chapter{Symmetric Group}

\end{document}